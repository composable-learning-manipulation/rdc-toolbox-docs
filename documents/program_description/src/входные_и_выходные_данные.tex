\newpage
\section{Входные и выходные данные}
Для программы предусмотрены несколько видов программ, описанных в руководстве оператора.

\subsection{Входные данные}
Входными данными для программы в зависимости от команды могут быть:
\begin{enumerate}
    \item [--] csv-файлы с записанными значениями положения $q$, скорости $\dot{q}$, ускорения $\ddot{q}$, моментов $\tau$, времени $t$ и др. данных,
	\item [--] urdf-файлы с кинематическим описанием робота или рабочего инструмента/нагрузки,
	\item [--] config-файлы с параметрами и настройками программы,
    \item [--] сериализованные модели обратной динамики робота в формате {.rdcm} (формат pickle),
    \item [--] {.np} файлы, содержащие в себе матрицу перестановок или редуцирования (формат numpy).
\end{enumerate}

\subsection{Выходные данные}
Выходными данными для программы в зависимости от режима могут быть:
\begin{enumerate}
    \item [--] csv-файлы с записанными значениями положения $q$, скорости $\dot{q}$, ускорения $\ddot{q}$, моментов $\tau$, времени $t$ и др. данных,
	\item [--] urdf-файлы с кинематическим описанием робота или рабочего инструмента/нагрузки,
	\item [--] config-файлы с параметрами и настройками программы,
    \item [--] сериализованные модели обратной динамики робота в формате {.rdcm} (формат pickle),
    \item [--] {.np} файлы, содержащие в себе матрицу перестановок или редуцирования (формат numpy),
    \item [--] графики сравнения текущего, желаемого или оценок значения моментов на сочленениях $\tau$ или пространственных сил и моментов $\begin{bmatrix} F_{xyz} & T_{xyz}\end{bmatrix}^T$.
\end{enumerate}