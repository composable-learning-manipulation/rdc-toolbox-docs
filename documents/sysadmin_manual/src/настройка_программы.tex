\newpage
\section{Настройка программы}

\subsection{Подготовка программы для сборки}
Сборка осуществляется в приложении <<Code Compose Studio>>. Для подготовки программы к сборке необходимо:
\begin{enumerate}
    \item перейти в окно <<Project>>$\to$<<Show build settings>>$\to$<<General>> (с помощью верхней панели управления);
    \item в секции <<Project type and tool-chain>> установить следующие параметры:
    \begin{itemize}
        \item[--] Compiler version: TI v22.6.1.LTS;
        \item[--] Output type: Executable;
        \item[--] Output format: eaby (ELF);
        \item[--] Device endianness: little.
    \end{itemize}
    \item перейти в окно <<Project>>$\to$<<Show build settings>>$\to$<<Build>>$\to$<<C2000 Compiler>>$\to$<<Processor Options>> и установить следующие параметры:
    \begin{itemize}
        \item[--] Processor version: 28;
        \item[--] Specify CLA support: cla2;
        \item[--] Specify floating point support: fpu32;
        \item[--] Specify support for enhanced integer division: idiv0;
        \item[--] Specify TMU support: tmu0;
        \item[--] Specify VCU support: vcrc.
    \end{itemize}
    \item перейти в окно <<Project>>$\to$<<Show build settings>>$\to$<<Build>>$\to$<<C2000 Compiler>>$\to$<<Predefined Symbols>> и объявить следующие переменные:
    \begin{itemize}
        \item[--] \_RAM;
        \item[--] DEVICE=f28838d.
    \end{itemize}
\end{enumerate}

\subsection{Настройка загруженной программы}

Настройка программы предполагает инициализацию и конфигурирование необходимых параметров объектов, при создании и использовании их в коде.
Для объекта $struct\ SelfDiagnosticParams$ таковыми параметрами являются:
\begin{enumerate}
    \item $\delta_{max}$ ($maxAirGap$)~--- максимальная величина зазора между статором и ротором,~м;
    \item $M_{max}$ ($maxHighFreqTorqueMagnitude$)~--- предельно допустимая амплитуда высокочастотных колебаний,~Н$\cdot$м;
    \item $w_{min}, w_{max}$ ($wMin$, $wMax$)~--- диапазон допустимых скоростей,~рад/c;
    \item Конструктивные параметры СДПМ:
    \begin{enumerate}
        \item $N$ ($coilNTurns$) -- приведенное число витков катушки сердечника статора;
        \item $\mu$ ($mu$) -- магнитная проницаемость магнитопровода,~Гн/м;
        \item $S_c$ ($statorCoreArea$) -- площадь сердечника статора,~м$^2$;
        \item $l$ ($statorWireLen$) -- длина проводника статора,~м;
        \item $S_m$ ($magnetCoreArea$) --  эффективная площадь элемента магнита,~м$^2$;
        \item $n_p$ ($n\_p$) -- число пар полюсов СДПМ;
    \end{enumerate}
    \item $dt$ -- период дискретизации входных данных.
\end{enumerate}

Входные данные представляют собой структуру $SelfDiagnosticInputState$ и заполняются другими программами, входящими в состав Программного обеспечения датчика положения в следующем виде:
\begin{enumerate}
    \item[--] $i_{\alpha, \beta}$ ($i\_ab$) -- мгновенные значения токов обмоток статора СДПМ,~А;
    \item[--] $\dot{i}_{\alpha, \beta}$ ($di\_ab$) -- мгновенные значения производных токов обмоток статора СДПМ,~А/с;
    \item[--] $u_{\alpha, \beta}$ ($u\_ab$) -- мгновенные значения напряжений обмоток статора СДПМ,~В;
    \item[--] $M$ ($magnet\_torque\_est$) -- электромагнитный момент двигателя,~Н$\cdot$м;
    \item[--] $\omega$ ($w$) -- скорость ротора,~рад/c;
    \item[--] $\Theta$ ($theta$) -- угловое положение ротора~рад;
    \item[--] $\hat{\omega}$ ($w\_est$) -- оценка скорости ротора,~рад/c.
\end{enumerate}
