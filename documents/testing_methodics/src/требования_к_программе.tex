\newpage
\section{Требования к программе}

\subsection{Проверка работоспособности \productname в соответствии с техническим заданием} \label{req:check-prog}

\subsubsection{Испытание работоспособности приема и обработки запросов на проверку} \label{req:functionality}
Критерий оценки: наличие в выводе программы тестирования ровно двух сообщений: <<\textbf{Finish. Execution complete}>>, наличие в конце вывода программы сообщения: <<\textbf{ОК}>>.

Пункт методики \ref{reqtest:functionality}.

\subsubsection{Испытание работоспособности функии проверки, получения и агрегации результатов} \label{req:scoring}
Критерий оценки: соответствие информации в поле <<score>> значению <<\textbf{0.0}>> и <<\textbf{100.0}>> в обоих тестах в любом порядке следования, наличие в конце вывода программы сообщения: <<\textbf{ОК}>>.

Пункт методики \ref{reqtest:scoring}.
  
\subsubsection{Испытание работоспособности функции загрузки решения и задач из репозиториев} \label{req:git_tools}
Критерий оценки: наличие в конце вывода программы сообщения: <<\textbf{ОК}>>.

Пункт методики \ref{reqtest:git_tools}.

\subsubsection{Испытание эффективности тестирующей системы при высокой пользовательской нагрузке} \label{req:load}
Критерий оценки: Линейная зависимость задержек в системе от числа запросов на проверку.

Пункт методики \ref{reqtest:load}.


\subsubsection{Испытание авторизации через тестовую почту} \label{req:check-self-email}
Необходимо наличие тестового email и следующих данных: тестовые логин и пароль от аккаунта.

Критерий оценки: наличие информации в поле <<Ваше имя на ресурсе>>. 

Пункт методики \ref{reqtest:pm-1}.

\subsubsection{Испытание авторизации через Gitlab} \label{req:check-gitlab-email}
Необходимо наличие тестового аккаунта на сайте gitlab.com.

Критерий оценки: наличие информации в поле <<Ваше имя на ресурсе>>. 

Пункт методики \ref{reqtest:pm-2}.

\subsubsection{Испытание авторизации через Github} \label{req:check-github-email}
Необходимо наличие тестового аккаунта на сайте github.com.

Критерий оценки: наличие информации в поле <<Ваше имя на ресурсе>>. 

Пункт методики \ref{reqtest:pm-3}.

\subsubsection{Испытание создания новых соревнований и задач} \label{req:check-competitions}
Наличие данных для авторизации на веб-сайте программы: логин (Username) и пароль (Password). Наличие входных данных для создания задачи: Dev repo, Dev token name, Dev token, Workflow path.

Критерий оценки: наличие записей в базе данных и на веб-сайте. 

Пункт методики \ref{reqtest:pm-4}.

\subsubsection{Испытание отправки алгоритма управления движением робота на тестирование} \label{req:check-sol-submit}
Необходимо наличие данных для входа в тестовый аккаунт: логин и пароль и данных для отправки алгоритма на проверку: Href to repository, Token Name, Token.

Критерий оценки: <<Наличие ненулевых значений в графе <<Active>> или <<Precessed>>. 

Пункт методики \ref{reqtest:pm-5}.
