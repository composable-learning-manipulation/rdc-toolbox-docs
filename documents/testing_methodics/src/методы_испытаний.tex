\newpage
\section{Методы испытаний}

\subsection{Настройка окружения для проведения тестирования} \label{reqtest:setup}
Для проведения испытаний программы \productname для для синтеза параметрических моделей обратной динамики робота, а также для осуществления идентификации их параметров, необходимо выполнить следующие действия:


\subsubsection{Подготовка окружения Anaconda}

Для подготовки окружения необходимо установить зависимости используемых библиотек в окружении Anaconda.

Для этого необходимо выполнить:
\begin{lstlisting}[language=bash, numbers=none, frame=single]
    conda env create -f environment.yml
\end{lstlisting}

Установить дополнительные зависимости:
\begin{lstlisting}[language=bash, numbers=none, frame=single]
    pip3 install -e 3rdparty/eagerpy
\end{lstlisting}

Или обновить установленное ранее окружение:
\begin{lstlisting}[language=bash, numbers=none, frame=single]
    conda env update -f environment.yml
\end{lstlisting}

Установить дополнительные зависимости:
\begin{lstlisting}[language=bash, numbers=none, frame=single]
    pip3 install -e 3rdparty/eagerpy
\end{lstlisting}

Или запустить подготовленное ранее окружение:
\begin{lstlisting}[language=bash, numbers=none, frame=single]
    conda activate rdc
\end{lstlisting}

\subsubsection{Установка библиотеки для управления роботом}

Выполнить установку библиотеки $z1\_sdk$ согласно инструкции в источнике.

\subsubsection{Сборка программного обеспечения}

Выполнить сборку программного обеспечения внутри окружения:
\begin{lstlisting}[language=bash, numbers=none, frame=single]
    pip3 install -e .
\end{lstlisting}

\subsection{Проведение испытаний программы \productname по п. \ref{req:functionality-calib} ПМ работоспособности модуля калибровки} \label{reqtest:functionality-calib}
Для проведения испытаний модуля калибровки программы \productname для оценивания положения центра и ориентации датчика силы в мировой системе координат, а также положения центра и ориентации наконечника рабочего инструмента в системе координат последнего звена робота оператор выполняет действия, указанные в п. \ref{reqtest:setup}, после чего делает следующее:
\begin{enumerate}
    \item Закрывает все открытые командные строки;
    \item Проверяет подключение робота;
    \item Запускает программы для калибровки согласно п. 3.1.1 и 3.1.2 Руководства оператора.
\end{enumerate}
 \programname считается выдержавшей проверку, если результат выведенный в командной строке соответсвует п. \ref{req:functionality-calib} ПМ.


\subsubsection{Проверка по п. \ref{req:functionality-gen} ПМ работоспособности модуля генерации} \label{reqtest:functionality-gen}

Для проведения испытаний модуля генерации программы \productname для генерации оптимальных траекторий на основе ряда Фурье, а также траекторий для идентиикации трения, оператор выполняет действия, указанные в п. \ref{reqtest:setup}, после чего делает следующее:
\begin{enumerate}
    \item Закрывает все открытые командные строки;
    % \item Проверяет подключение робота
    \item Запускает программы для генерации согласно п. 3.3.1 и 3.3.2 Руководства оператора.
\end{enumerate}
\programname считается выдержавшей проверку, если результат выведенный в командной строке соответсвует п. \ref{req:functionality-gen} ПМ.

\subsubsection{Проверка по п. \ref{req:functionality-exec} ПМ работоспособности модуля исполнения} \label{reqtest:functionality-exec}
Для проведения испытаний модуля исполнения программы \productname для исполнения предварительно полученной траектории роботом-манипулятором в физическом мире или посредством симуляционной среды, оператор выполняет действия, указанные в п. \ref{reqtest:setup}, после чего делает следующее:
\begin{enumerate}
    \item Закрывает все открытые командные строки;
    \item Проверяет подключение робота или готовность симуляционной среды;
    \item Запускает программу для исполнения траекторий согласно п. 3.2.1 Руководства оператора.
\end{enumerate}
\programname считается выдержавшей проверку, если результат выведенный в командной строке соответсвует п. \ref{req:functionality-exec} ПМ.

\subsubsection{Проверка по п. \ref{req:functionality-ident} ПМ  работоспособности модуля идентификации модели обратной динамики} \label{reqtest:functionality-ident}
Для проведения испытаний модуля идентификации программы \productname для идентификации модели обратной динамики оператор выполняет действия, указанные в п. \ref{reqtest:setup}, после чего делает следующее:
\begin{enumerate}
    \item Закрывает все открытые командные строки;
    \item Запускает программу для выполнения идентиикации модели трения и полной модели обратной динамики согласно п. 3.4.1 и п. 3.4.2 Руководства оператора.
\end{enumerate}
\programname считается выдержавшей проверку, если результат выведенный в командной строке соответсвует п. \ref{req:functionality-exec} ПМ.

\subsection{Проверка по п. \ref{req:functionality-verif} ПМ  работоспособности модуля верификации модели обратной динамики} \label{reqtest:functionality-verif}
Для проведения испытаний модуля верификации программы \productname для верификации модели обратной динамики методом вычисленных моментов
или по оценке пространственных сил и моментов на рабочем инструменте оператор выполняет действия, указанные в п. \ref{reqtest:setup}, после чего делает следующее:
\begin{enumerate}
  \item Закрывает все открытые командные строки;
  \item Проверяет подключение робота и корректность выставленных углов наклонной платформы;
  \item Запускает программу для выполнения идентиикации модели трения и полной модели обратной динамики согласно п. 3.5.1 и п. 3.5.2 Руководства оператора.
\end{enumerate}


\newpage
\section{Отчетность}

\subsection{Оформление результатов испытаний}
Результаты испытаний по каждому пункту требований к программе оформляются протоколами, представляемыми на заседании комиссии. Типовая форма протокола испытаний приведена в приложении А.

В протоколы вносятся все первичные данные, получаемые в процессе испытаний.

Большой объем первичных данных (в том числе, распечаток ЭВМ, содержащих однозначную привязку к месту, времени и объекту испытаний) допускается приводить в приложениях к протоколам, подписываемых лицами, проводящими испытания.

По результатам испытаний в течение 3 дней комиссией составляется акт испытаний.

Акт испытаний должен содержать:
\begin{enumerate}
    \item подтверждение выполнения программы испытаний;
    \item оценку результатов испытаний с конкретными точными формулировками, отражающими соответствие испытуемого комплекса требованиям ТЗ;
    \item выводы по результатам испытаний.
\end{enumerate}

К акту прилагаются протоколы испытаний по пунктам программы.

Первичные материалы испытаний хранятся у Исполнителя.
