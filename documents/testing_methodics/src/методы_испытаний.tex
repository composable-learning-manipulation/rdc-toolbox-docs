\newpage
\section{Методы испытаний}

\subsection{Настройка окружения для проведения тестирования} \label{reqtest:setup}
Для проведения испытаний программы \productname\ для организации удаленной разработки и тестирования моделей и алгоритмов планирования и управления движением роботов необходимо выполнить следующие действия:


\subsubsection{Подготовка кластера Kubernetes}

Для развертывания приложения необходимо выполнить предварительную подготовку кластера Kubernetes microk8s.

Активируйте следующие плагины для microk8s:
\begin{lstlisting}[language=bash, numbers=none, frame=single]
    microk8s enable dns hostpath-storage ingress metallb
\end{lstlisting}
Для metallb выберите диапазон адресов: 192.168.55.105-192158.55.120

Обновите kube config:
\begin{lstlisting}[language=bash, numbers=none, frame=single]
    microk8s config > ~/.kube/config
\end{lstlisting}

Создайте пространство имен для kubectl:
\begin{lstlisting}[language=bash, numbers=none, frame=single]
    kubectl create namespace rf-checker-test
    kubectl create namespace rf-website-test
\end{lstlisting}

Создайте imagePullSecret в gitlab, как показано ниже, для скачивания из приватного реестра образов:
\begin{lstlisting}[language=bash, numbers=none, frame=single]
    kubectl create secret docker-registry regcred -n rf-checker-test --docker-server=https://registry.gitlab.com --docker-username=<TOKEN NAME| USERNAME> --docker-password=<TOKEN> 
    kubectl create secret docker-registry regcred -n rf-website-test --docker-server=https://registry.gitlab.com --docker-username=<TOKEN NAME| USERNAME> --docker-password=<TOKEN> 
\end{lstlisting}
А также выполните авторизацию для werf:
\begin{lstlisting}[language=bash, numbers=none, frame=single]
    werf cr login registry.gitlab.com/roboforces-itmo/platform/ --username <TOKEN_NAME | USERNAME> --password <TOKEN>
\end{lstlisting}

Предварительно токен нужно создать в сервисе, который используется для ведения рееста образов.


Добавьте ярлык для ноды вашего Kubernetes кластера:
\begin{lstlisting}[language=bash, numbers=none, frame=single]
    kubectl label node <NODE_NAME> power=high
    kubectl label node <NODE_NAME> worker=main
    kubectl label node <NODE_NAME> server=web
\end{lstlisting}

NODE\_NAME, т.е имя ноды может быть получено путем вызова команды:
\begin{lstlisting}[language=bash, numbers=none, frame=single]
    kubectl get nodes
\end{lstlisting}

Создайте необходимые CRDs:
\begin{lstlisting}[language=bash, numbers=none, frame=single]
    microk8s kubectl apply -f crds/argo-crds-namespaced.yaml
\end{lstlisting}




\subsubsection{Запуск модуля rf-checker программы \productname}
Для развертывания модуля установите его с помощью werf:
\begin{lstlisting}[language=bash, numbers=none, frame=single]
    werf converge --repo registry.gitlab.com/roboforces-itmo/platform/rf-checker --set replicaCount=0 --set tenant.enabled=true --dev --env test
\end{lstlisting}
При этом также в кластере будут автоматически установлены redis и argo-workflow Helm-чарты, необходимы для работы моодуля rf-checker \productname. 
Для запуска при тестировании рекомендуется использовать Docker compose, путем вызова:
\begin{lstlisting}[language=bash, numbers=none, frame=single]
    docker compose up --build
\end{lstlisting}

С этого момента программа считается запущенной. Коммуникация с программой может быть осужествлена посредством отправки запросов в Redis очередь \textbf{rf-queue}. Для подключения к Redis, необходимо знать IP адрес и порт подключения. Чтобы их узнать можно воспользоваться инструментом kubectl, входящим в состав Kubernetes microk8s:
\begin{lstlisting}[language=bash, numbers=none, frame=single]
    microk8s.kubectl get all . -n rf-checker-test
\end{lstlisting}
Пароль для подключения определяется в файле .local.env файла на момент его создания.

\subsubsection{Запуск модуля rf-website программы \productname}
Для развертывания модуля установите его с помощью werf:
\begin{lstlisting}[language=bash, numbers=none, frame=single]
    werf converge --dev --repo "registry.gitlab.com/roboforces-itmo/platform/rf-website" --env test --set redis.enabled=false
\end{lstlisting}
При этом также в кластере будут автоматически установлены mongodb и Postgres Helm-чарты, необходимы для работы модуля rf-website \productname. 
С этого момента программа считается запущенной. Для подключения к сайту, необходимо знать IP адрес и порт подключения. Чтобы их узнать можно воспользоваться инструментом kubectl, входящим в состав Kubernetes microk8s:
\begin{lstlisting}[language=bash, numbers=none, frame=single]
    microk8s.kubectl get all . -n rf
\end{lstlisting}

Чтобы сделать программу доступной по IP хоста, необходимо ввести команду для открытия порта:
\begin{lstlisting}[language=bash, numbers=none]
    microk8s.kubectl port-forward service/rf-website-test 8000:8080 -n rf-website-test
\end{lstlisting}

При запуске веб-страницы программы (\url{http://127.0.0.1:8000}) выводится подробная информация о содержании и возможностях клиентского модуля. В случае возникновения любых проблем при работе с программой клиентского модуля, веб-страничка отобразит информацию об обшибке или успешном выполнении.


\subsection{Проведение испытаний программы \productname\ по п. \ref{req:functionality} ПМ приема и обработки запросов на проверку} \label{reqtest:functionality}
Для проведения испытаний модуля rf-checker  программы \productname\ для организации удаленной разработки и тестирования моделей и алгоритмов планирования и управления движением роботов оператор выполняет действия, указанные в п. \ref{reqtest:setup}, после чего делает следующее:
\begin{enumerate}
    \item Закрывает все открытые командные строки
    \item Оператор актуализирует информацию в конфигурационном файле, об используемом Kubernetes кластере:
    \begin{lstlisting}[language=bash, numbers=none, frame=single]
        microk8s config > ~/.kube/config
    \end{lstlisting}
    \item В другой консоли оператор переходит с помощью командной строки, открытой в корне директории программы в директорию $checking\_worker$:
    \begin{lstlisting}[language=bash, numbers=none, frame=single]
    cd checking_worker
    \end{lstlisting}
    \item Оператор запускает программу для тестирования:
    \begin{lstlisting}[language=bash, numbers=none, frame=single]
    python3 tests/test_send_simple_python_task.py
    \end{lstlisting}
\end{enumerate}
Программа \programname\ считается выдержавшей проверку, если результат выведенный в командной строке соответсвует п. \ref{req:functionality} ПМ.


\subsubsection{Проверка по п. \ref{req:scoring} ПМ получения и агрегации результатов} \label{reqtest:scoring}

Для проведения испытаний модуля rf-checker  программы \productname\ для организации удаленной разработки и тестирования моделей и алгоритмов планирования и управления движением роботов оператор выполняет действия, указанные в п. \ref{reqtest:setup}, после чего делает следующее:
\begin{enumerate}
    \item Закрывает все открытые командные строки
    \item Оператор актуализирует информацию в конфигурационном файле, об используемом Kubernetes кластере:
    \begin{lstlisting}[language=bash, numbers=none, frame=single]
        microk8s config > ~/.kube/config
    \end{lstlisting}
    \item В другой консоли оператор переходит с помощью командной строки, открытой в корне директории программы в директорию $checking\_worker$:
    \begin{lstlisting}[language=bash, numbers=none, frame=single]
    cd checking_worker
    \end{lstlisting}
    \item Оператор запускает программу для тестирования:
    \begin{lstlisting}[language=bash, numbers=none, frame=single]
    python3 tests/test_send_checking_pipline.py
    \end{lstlisting}
\end{enumerate}
Программа \programname\ считается выдержавшей проверку, если результат выведенный в командной строке соответсвует п. \ref{req:scoring} ПМ.

\subsubsection{Проверка по п. \ref{req:git_tools} ПМ загрузки решения и задач из репозиториев} \label{reqtest:git_tools}
Для проведения испытаний модуля rf-checker платформы \productname\ для организации удаленной разработки и тестирования моделей и алгоритмов планирования и управления движением роботов оператор выполняет действия, указанные в п. \ref{reqtest:setup}, после чего делает следующее:
\begin{enumerate}
    \item Закрывает все открытые командные строки
    \item Оператор актуализирует информацию в конфигурационном файле, об используемом Kubernetes кластере:
    \begin{lstlisting}[language=bash, numbers=none, frame=single]
        microk8s config > ~/.kube/config
    \end{lstlisting}
    \item В другой консоли оператор переходит с помощью командной строки, открытой в корне директории программы в директорию $checking\_worker$:
    \begin{lstlisting}[language=bash, numbers=none, frame=single]
    cd checking_worker
    \end{lstlisting}
    \item Оператор запускает программу для тестирования:
    \begin{lstlisting}[language=bash, numbers=none, frame=single]
    python3 tests/test_git_tools.py
    \end{lstlisting}
\end{enumerate}
Программа \programname\ считается выдержавшей проверку, если результат выведенный в командной строке соответсвует п. \ref{req:git_tools} ПМ.

\subsubsection{Проверка по п. \ref{req:load} ПМ эффективности при высокой пользовательской нагрузке} \label{reqtest:load}
Для проведения испытаний модуля rf-checker платформы \productname\ для организации удаленной разработки и тестирования моделей и алгоритмов планирования и управления движением роботов оператор выполняет действия, указанные в п. \ref{reqtest:setup}, после чего делает следующее:
\begin{enumerate}
    \item Закрывает все открытые командные строки
    \item Оператор актуализирует информацию в конфигурационном файле, об используемом Kubernetes кластере:
    \begin{lstlisting}[language=bash, numbers=none, frame=single]
        microk8s config > ~/.kube/config
    \end{lstlisting}
    \item В другой консоли оператор переходит с помощью командной строки, открытой в корне директории программы в директорию $checking\_worker$:
    \begin{lstlisting}[language=bash, numbers=none, frame=single]
    cd checking_worker
    \end{lstlisting}
    \item Оператор запускает программу для нагрузочного тестирования:
    \begin{lstlisting}[language=bash, numbers=none, frame=single]
    python3 tests/test_load.py
    \end{lstlisting}
\end{enumerate}
Программа \programname\ считается выдержавшей проверку, если результат представленный на графике нагрузочной характеристики соответсвует п. \ref{req:load} ПМ.

\subsection{Проверка по п. \ref{req:check-self-email} ПМ авторизации через тестовую почту} \label{reqtest:pm-1}
Для проведения испытаний модуля rf-website платформы \productname\ необходимо выполнить подготовительные действия.

Оператор выполняет действия, указанные в Руководстве оператора, после чего делает сделующее:
\begin{enumerate}
    \item Оператор открывает веб-браузер, вводит в строке следующий адрес платформы веб-интерфейса программы \url{http://127.0.0.1:8000/}, см. \ref{fig:url_enter}, и нажимает клавишу Enter↵,. \label{it: start-page}
    \illustration[][Пример ввода url.][0.9]{pmi_7_3/1.url_enter}[fig:url_enter]

    \item Оператор разворачивает меню слева, нажав на стрелку в верхнем левом углу и переходит во вкладку <<Личный кабинет>, см \ref{fig:lk}. \label{it: lk}
    \illustration[][Переход в личный кабинет.][0.3]{pmi_7_3/3.lk}[fig:lk]

    \item Затем оператор переходит по ссылке <<Зарегистрируйтесь сейчас>>.
    \illustration[][Указатель на ссылку <<Зарегистрируйтесь сейчас>>.][0.3]{pmi_7_3/4.sign_up_link}[fig:signup-link]

    \item Оператор вводит заготовленные логин и дважды пароль, затем клавишу <<Зарегистрироваться>> \ref{fig:link_signup}. Оператор должен увидеть результат как на \ref{fig:ask_confirm_email}.
    {
        \centering
        \begin{tabular}[c]{ m{0.5\textwidth} m{0.5\textwidth} }		
            {
                \begin{minipage}[t]{0.45\textwidth}
                    \centering
                    \illustration[][Указатель на ссылку <<Зарегистрироваться>>.][0.9]{pmi_7_3/5.lk_self_data}[fig:link_signup]
                \end{minipage}
            } & {
                \begin{minipage}[t]{0.45\textwidth}
                    \centering
                    \illustration[][Сообщение о подтверждении адреса.][0.9]{pmi_7_3/6.email_resul}[fig:ask_confirm_email]
                \end{minipage}
            } \\		
        \end{tabular}
    }

    \item Оператор открывает почту, находит письмо от платформы roboforces и переходит по ссылке для подтверждения почты \ref{fig:confirm_msg_email}. После перехода по ссылке оператор нажимает на кнопку <<Подтвердить>> \ref{fig:email_confirm}. \label{it: email-accout-data}
    {
        \centering
        \begin{tabular}[c]{ m{0.5\textwidth} m{0.5\textwidth} }		
            {
                \begin{minipage}[t]{0.45\textwidth}
                    \centering
                    \illustration[][Сообщение по подтверждению почты.][0.9]{pmi_7_3/7.email_check}[fig:confirm_msg_email]
                \end{minipage}
            } & {
                \begin{minipage}[t]{0.45\textwidth}
                    \centering
                    \illustration[][Сообщение о подтверждении адреса.][0.9]{pmi_7_3/6.1.email_confirm}[fig:email_confirm]
                \end{minipage}
            } \\		
        \end{tabular}
    }

    \item Оператор возвращается на начальную страничку веб-интерфейса \url{http://127.0.0.1:8000/} как в п. \ref{it: start-page} инструкции, открывает вкладку <<Личный кабинет>>, аналогично п. \ref{it: lk} инструкции, вводит email и пароль из п. \ref{it: email-accout-data} инструкции, далее нажимает кнопку  <<Войти>>, см. \ref{fig:enter_acc}.
    \illustration[][Страница входа в аккаунт.][0.5]{pmi_7_3/8.enter_account}[fig:enter_acc]

    \item Оператор открывает вкладку <<Личный кабинет>>, аналогично п. \ref{it: lk} инструкции, и проверяет наличие информации в поле <<Ваше имя на ресурсе>> (см \ref{fig:nikname}).
    \illustration[][Страница личного кабинета.][0.5]{pmi_7_3/9.nikname}[fig:nikname]
\end{enumerate}


\subsection{Проверка по п. \ref{req:check-gitlab-email} ПМ авторизации через Gitlab} \label{reqtest:pm-2}
Для проведения испытаний модуля rf-website платформы \productname\ необходимо выполнить подготовительные действия.

Оператор должен предварительно авторизоваться на веб-сайте \url{gitlab.com}

Оператор выполняет действия, указанные в пункте~\ref{reqtest:setup}, после чего делает сделующее:
\begin{enumerate}
    \item Оператор открывает новую вкладку в браузере и выполняет п. \ref{it: start-page} - \ref{it: lk} раздела \ref{reqtest:pm-1}.
    \item Оператор нажимает иконку Gitlab \ref{fig:gitlab_auth_icon}. Оператора перенесет по ссылке, где он нажимает на кнопку <<Authorize>> \ref{fig:gitlab_email_confirm}.
    {
        \centering
        \begin{tabular}[c]{ m{0.5\textwidth} m{0.5\textwidth} }		
            {
                \begin{minipage}[t]{0.45\textwidth}
                    \centering
                    \illustration[][Указатель на иконку gitlab.][0.9]{pmi_7_4/1.Auth}[fig:gitlab_auth_icon]
                \end{minipage}
            } & {
                \begin{minipage}[t]{0.45\textwidth}
                    \centering
                    \illustration[][Подтверждение авторизации через gitlab.][0.9]{pmi_7_4/4.auth_gitlab_access}[fig:gitlab_email_confirm]
                \end{minipage}
            } \\		
        \end{tabular}
    }

    \item Оператор переходит на почту от аккаунта Github и подтверждает ее, аналогично п. \ref{it: email-accout-data} раздела \ref{reqtest:pm-1}.
    \item Оператор возвращается на начальную страницу \url{http://127.0.0.1:8000/} как в п. \ref{it: start-page}, открывает вкладку <<Личный кабинет>>, нажимает на иконку Gitlab и нажимает на кнопку <<Продолжить>>, чтобы авторизоваться. После этого оператора автоматически перенесет на стартовую страницу веб-интерфейса.
    \item Оператор открывает вкладку <<Личный кабинет>>, аналогично п. \ref{it: lk} инструкции раздела \ref{reqtest:pm-1}, и проверяет наличие информации в поле <<Ваше имя на ресурсе>>.
\end{enumerate}

\subsection{Проверка по п. \ref{req:check-github-email} ПМ авторизации через Github} \label{reqtest:pm-3}
Для проведения испытаний модуля rf-website платформы \productname\ необходимо выполнить подготовительные действия.

Оператор должен предварительно авторизоваться на веб-сайте \url{github.com}

Оператор выполняет действия, указанные в пункте~\ref{reqtest:setup}, после чего делает сделующее:
\begin{enumerate}
    \item Оператор открывает новую вкладку в браузере и выполняет п. \ref{it: start-page} - \ref{it: lk} раздела \ref{reqtest:pm-1}.
    \item Оператор нажимает иконку Github \ref{fig:github_auth_icon}, далее его перенесет по ссылке, где он нажимает на кнопку <<Authorize roboforces-itmo>> \ref{fig:github_email_confirm}.
    {
        \centering
        \begin{tabular}[c]{ m{0.5\textwidth} m{0.5\textwidth} }		
            {
                \begin{minipage}[t]{0.45\textwidth}
                    \centering
                    \illustration[][Указатель на иконку github.][0.9]{pmi_7_5/1.Auth}[fig:github_auth_icon]
                \end{minipage}
            } & {
                \begin{minipage}[t]{0.45\textwidth}
                    \centering
                    \illustration[][Подтверждение авторизации через github.][0.9]{pmi_7_5/3.github_email_verfication}[fig:github_email_confirm]
                \end{minipage}
            } \\		
        \end{tabular}
    }

    \item Оператор переходит на почту от аккаунта Github и подтверждает ее, аналогично п. \ref{it: email-accout-data} раздела \ref{reqtest:pm-1}.
    \item Оператор возвращается на начальную страницу \url{http://127.0.0.1:8000/} как в п. \ref{it: start-page}, открывает вкладку <<Личный кабинет>>, нажимает на иконку Github и нажимает на кнопку <<Продолжить>>, чтобы авторизоваться. После этого оператора автоматически перенесет на стартовую страницу веб-интерфейса.
    \item Оператор открывает вкладку <<Личный кабинет>>, аналогично п. \ref{it: lk} инструкции раздела \ref{reqtest:pm-1}, и проверяет наличие информации в поле <<Ваше имя на ресурсе>>.
\end{enumerate}

\subsection{Проверка по п. \ref{req:check-competitions} ПМ создание новых соревнований и задач} \label{reqtest:pm-4}
Для проведения испытаний модуля rf-website платформы \productname\ необходимо выполнить подготовительные действия.

Оператор выполняет действия, указанные в пункте~\ref{reqtest:setup}, после чего делает сделующее:
\begin{enumerate}
    \item Для получения доступа к базе данных оператор получает данные для авторизации от разработчика (например, Username: admin, Password: admin).
    \item Оператор открывает веб-браузер и вводит в строке следующий адрес веб-интерфейса программы \url{http://127.0.0.1:8000/admin}. Далее вводит в поля <<Username>> и <<Password>> данные для авторизации из предыдущего пункта инструкции. В результате оператор должен увидеть страницу администратора с таблицами <<ACCOUNTS>>, <<AUTHENTICATION AND AUTHORIZATION>>, <<CONTESTS>>, <<SITES>> и др., как на рисунке ниже.
    \illustration[][Страница администратора веб интерфейса.][0.5]{pmi_7_6/1.admin_view}[fig:admin_veiw]

    \item Чтобы добавить новое соревнование оператор находит таблицу <<CONTESTS>> и кликает по полю <<Competitions>> \ref{fig:competition_table}, чтобы перейти на страницу создания соревнования. Далее оператор нажимает на кнопку <<ADD COMPETITION +>> \ref{fig:add_comp_btn}.
    {
        \centering
        \begin{tabular}[c]{ m{0.5\textwidth} m{0.5\textwidth} }		
            {
                \begin{minipage}[t]{0.45\textwidth}
                    \centering
                    \illustration[][Вид таблицы <<Competitions>>.][0.9]{pmi_7_6/2.Compettions}[fig:competition_table]
                \end{minipage}
            } & {
                \begin{minipage}[t]{0.45\textwidth}
                    \centering
                    \illustration[][Указатель на кнопку <<ADD COMPETITION +>>.][0.9]{pmi_7_6/3.add_comp}[fig:add_comp_btn]
                \end{minipage}
            } \\		
        \end{tabular}
    }

    \item Оператор заполняет информацию о соревновании, пример указан на рисунках ниже. Важно загрузить изображение в поле <<Image>> и заполнить поле <<Description>>.
    {
        \centering
        \begin{tabular}[c]{ m{0.5\textwidth} m{0.5\textwidth} }		
            {
                \begin{minipage}[t]{0.45\textwidth}
                    \centering
                    \illustration[][Форма для создания соревнования 1.][0.9]{pmi_7_6/4.1.info}[fig:comp_form_1]
                \end{minipage}
            } & {
                \begin{minipage}[t]{0.45\textwidth}
                    \centering
                    \illustration[][Форма для создания соревнования 2.][0.9]{pmi_7_6/4.2.info}[fig:comp_form_2]
                \end{minipage}
            } \\		
        \end{tabular}
    }

    \item После завершения заполнения всех полей оператор нажимает на клавишу <<Save>>, после чего его перенесет обратно к полю <<Competitions>>, где он должен увидеть, что появилось новая запись в базе данных с указанным в предыдущем пункте инструкции названием соревнования см. \ref{fig:finish_comp_fill}.
    \illustration[][Верификация создания нового соревнования.][0.5]{pmi_7_6/5.Finish_comptition}[fig:finish_comp_fill]
    Финально, оператор возвращается на стартовую страницу панели администратора, нажав по ссылке <<Home>> в левом верхнем углу, что указывает стрелка на рисунке выше.

    \item Чтобы добавить новую задачу к соревнованию оператор находит таблицу <<CONTESTS>> и кликает по полю <<Tasks>>, чтобы перейти на страницу создания соревнования. Далее оператор нажимает на кнопку <<ADD TASK +>>.
    \item Далее оператор должен получить данные (Dev repo, Dev token name, Dev token, Workflow path) от разработчика и ввести их в соотвествующие поля.
    \item Оператор заполняет все поля в форме и выбирает <<Competition object>> в поле <<Competition>>, см пример рисунках ниже.
    {
        \centering
        \begin{tabular}[c]{ m{0.5\textwidth} m{0.5\textwidth} }		
            {
                \begin{minipage}[t]{0.45\textwidth}
                    \centering
                    \illustration[][Форма для создания задачи 1.][0.9]{pmi_7_6/7.task_info_1}[fig:task_form_1]
                \end{minipage}
            } & {
                \begin{minipage}[t]{0.45\textwidth}
                    \centering
                    \illustration[][Форма для создания задачи 2.][0.9]{pmi_7_6/8.task_info_2}[fig:task_form_2]
                \end{minipage}
            } \\		
        \end{tabular}
    }

    \item После завершения заполнения всех полей оператор нажимает на клавишу <<Save>>, после чего его перенесет обратно к полю <<Competitions>>, где он должен увидеть, что появилось новая запись в базе данных с указанным в предыдущем пункте инструкции названием задачи см. \ref{fig:save_task}.
    \illustration[][Верификация создания новой задачи.][0.5]{pmi_7_6/9.task_save}[fig:save_task]
\end{enumerate}

\subsection{Проверка по п. \ref{req:check-sol-submit} ПМ отправки алгоритма управления движением робота на тестирование} \label{reqtest:pm-5}
Для проведения испытаний модуля rf-website платформы \productname\ необходимо выполнить подготовительные действия.

Оператор выполняет действия, указанные в пункте~\ref{reqtest:setup}, после чего делает сделующее:
\begin{enumerate}
    \item Оператор открывает терминал и вводит команду:
    \begin{lstlisting}[language=bash, numbers=none]
        celery --broker=redis://:testo-mesto@127.0.0.1:6379// flower
    \end{lstlisting}
    \item Для получения доступа к веб-интерфесу программы оператор получает данные для авторизации от разработчика (например, Username: test-user@gmail.com, Password: test-user).
    \item Оператор открывает новую вкладку в браузере и выполняет п. \ref{it: start-page} - \ref{it: lk} раздела \ref{reqtest:pm-1}. Далее вводит в поля <<Username>> и <<Password>> данные для авторизации из предыдущего пункта инструкции и нажимает кнопку <<Войти>>.
    \item После входа оператор должен увидеть соревнования, которые были добавлены в разделе \ref{reqtest:pm-4}. Оператор должен нажать на кнопку <<Перейти к соревнованию>>, где он увидит задачу. Оператор нажимает на кнопку <<Перейти к задаче>>.
    \item Разработчик должен выдать оператору следующую дополнительную информацию (Href to repository и из раздела Repo token два поля Token Name, Token), которую вводит в соответветствующие поля.
    \item Оператор нажимает на кнопку <<Отправить на проверку>>.
    \item Чтобы проверить, что задача успешно была отправлена на проверку, оператор должен открыть новую страницу в веб-браузере \url{http://127.0.0.1:5555/}. И увидеть ненулевое значение в графе <<Active>> или <<Precessed>>, см. \ref{fig:load_task_verif}.
    \illustration[][Верификация загрузки алгоритма на проверку.][0.9]{pmi_7_7/1.flower_result}[fig:load_task_verif]
\end{enumerate}


\newpage
\section{Отчетность}

\subsection{Оформление результатов испытаний}
Результаты испытаний по каждому пункту требований к программе оформляются протоколами, представляемыми на заседании комиссии. Типовая форма протокола испытаний приведена в приложении А.

В протоколы вносятся все первичные данные, получаемые в процессе испытаний.

Большой объем первичных данных (в том числе, распечаток ЭВМ, содержащих однозначную привязку к месту, времени и объекту испытаний) допускается приводить в приложениях к протоколам, подписываемых лицами, проводящими испытания.

По результатам испытаний в течение 3 дней комиссией составляется акт испытаний.

Акт испытаний должен содержать:
\begin{enumerate}
    \item подтверждение выполнения программы испытаний;
    \item оценку результатов испытаний с конкретными точными формулировками, отражающими соответствие испытуемого комплекса требованиям ТЗ;
    \item выводы по результатам испытаний.
\end{enumerate}

К акту прилагаются протоколы испытаний по пунктам программы.

Первичные материалы испытаний хранятся у Исполнителя.

