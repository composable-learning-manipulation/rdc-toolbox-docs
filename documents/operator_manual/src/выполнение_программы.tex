\newpage
\section{Выполнение программы}
% Раздел <<Выполнение программы>> содержит сведения о последовательности действий оператора, обеспечивающих загрузку, запуск, выполнение и завершение программы. Приведено описание функций, формата и возможных вариантов команд, с помощью которых оператор осуществляет загрузки и управляет выполнением программы, а также ответы программы на эти команды.

Взаимодействие оператора и программы обеспечивается посредством интерфейса командной строки или консоли.
Для всех команд программы предусмотрена форма интерактивного взаимедйствия, где оператору необходимо ввести или входные данные или подтвердить дальнейшее выполнение программы.

Ниже представлена структурированная документация по командам для вызова и их параметрам.

\textbf{Использование:}
\begin{lstlisting}[language=python, numbers=none, frame=single]
    rdc [OPTIONS] COMMAND [ARGS]...
\end{lstlisting}

\textbf{Параметры:}
\begin{center}
\fontsize{10pt}{10pt}\selectfont
\begin{longtable}[]{p{5cm}|p{2cm}|p{3.5cm}|p{5cm}}
    \hline
\toprule()
Параметр & Тип & Описание & По умолчанию \\
\hline
\midrule()
\endhead
\texttt{-\/-help} & boolean & показать справку &
\texttt{False} \\
\bottomrule()
\hline
\end{longtable}
\end{center}

\hypertarget{rdc-calib}{%
\subsection{Команда rdc calib}\label{rdc-calib}}

Калибровка СК датчика силы/инструмента.

Команду следует использовать перед проверкой rdcm модели.

\textbf{Использование:}
\begin{lstlisting}[language=python, numbers=none, frame=single]
rdc calib [OPTIONS] COMMAND [ARGS]...
\end{lstlisting}

\textbf{Параметры:}
\begin{center}
\fontsize{10pt}{10pt}\selectfont
\begin{longtable}[]{p{5cm}|p{2cm}|p{3.5cm}|p{5cm}}
    \hline
\toprule()
Параметр & Тип & Описание & По умолчанию \\
\hline
\midrule()
\endhead
\texttt{-\/-help} & boolean & показать справку &
\texttt{False} \\
\bottomrule()
\hline
\end{longtable}
\end{center}

\hypertarget{rdc-calib-ftsensor}{%
\subsubsection{ Команда rdc calib ftsensor}\label{rdc-calib-ftsensor}}

Калибровка СК датчика силы.

Команда используется для оценивания положения центра и ориентации датчика силы в мировой системе координат.

\textbf{Использование:}
\begin{lstlisting}[language=python, numbers=none, frame=single]
rdc calib ftsensor [OPTIONS]
\end{lstlisting}

\textbf{Параметры:}
\begin{center}
\fontsize{10pt}{10pt}\selectfont
\begin{longtable}[]{p{5.2cm}|p{2cm}|p{3.5cm}|p{5cm}}
    \hline
\toprule()
Параметр & Тип & Описание & По умолчанию \\
\hline
\midrule()
\endhead
\texttt{-\/-urdf\_z1} & text & путь до urdf робота &
\texttt{./urdfs/z1\_ee.urdf} \\
\hline
\texttt{-\/-urdf\_z1\_link\_name} & text & название СК конца рабочего инструмента &
\texttt{link\_ee} \\
\hline
\texttt{-\/-urdf\_tool} & text & путь до urdf рабочего инструмента &
\texttt{./urdfs/tool\_calib.urdf} \\
\hline
\texttt{-\/-urdf\_tool\_link\_name} & text & название СК основания рабочего инструмента & \texttt{tool\_start\_link} \\
\hline
\texttt{-\/-urdf\_sensor} & text & путь до результирующей urdf датчика силы &
\texttt{./urdfs/sensor.urdf} \\
\hline
\texttt{-\/-pitch\_deg} & float & угол наклона платформы вокруг оси x &
\texttt{0} \\
\hline
\texttt{-\/-roll\_deg} & float & угол наклона платформы вокруг оси y &
\texttt{0} \\
\hline
\texttt{-\/-synth} & boolean & синтетические данные & \texttt{False} \\
\hline
\texttt{-\/-help} & boolean & показать справку &
\texttt{False} \\
\hline
\bottomrule()
\hline
\end{longtable}
\end{center}

\hypertarget{rdc-calib-tool}{%
\subsubsection{ Команда rdc calib tool}\label{rdc-calib-tool}}

Калибровка СК рабочего инструмента робота.

Команда может использоваться для оценивания координат и ориентации наконечника рабочего инструмента в системе координат последнего звена робота.

\textbf{Использование:}
\begin{lstlisting}[language=python, numbers=none, frame=single]
rdc calib tool [OPTIONS]
\end{lstlisting}

\textbf{Параметры:}
\begin{center}
\fontsize{10pt}{10pt}\selectfont
\begin{longtable}[]{p{4.5cm}|p{2cm}|p{4.0cm}|p{5.2cm}}
    \hline
\toprule()
Параметр & Тип & Описание & По умолчанию \\
\hline
\midrule()
\endhead
\texttt{-\/-urdf\_robot} & text & путь до urdf робота &
\texttt{./urdfs/z1\_ee.urdf} \\
\hline
\texttt{-\/-urdf\_tool} & text & путь до urdf рабочего инструмента &
\texttt{./urdfs/tool.urdf} \\
\hline
\texttt{-\/-fixed\_joint\_name} & text & имя сочленения для присоединения рабочего инструмента &
\texttt{fixed\_tool\_joint} \\
\hline
\texttt{-\/-out\_urdf} & text & путь до urdf откалиброванного рабочего инструмента &
\texttt{./urdfs/tool\_calib.urdf} \\
\hline
\texttt{-\/-load\_data} & text & путь до данных для калибровки & None \\
\hline
\texttt{-\/-num\_poses} & integer & количество  точек для калибровки
(мин.:3, рекомм.:30) & \texttt{30} \\
\hline
\texttt{-\/-assume\_coaxial} & boolean & предполагать что рабочий инструмент соосен последнему звену & \texttt{False} \\
\hline
\texttt{-\/-help} & boolean & показать справку &
\texttt{False} \\
\bottomrule()
\hline
\end{longtable}
\end{center}

\hypertarget{rdc-exec}{%
\subsection{Команда rdc exec}\label{rdc-exec}}

Выполнить на работе предоставленную в csv траекторию

Проверить траекторию на самостолкновения и столкновения с полом, а затем выполнить ее используя Z1RobotApi.

\textbf{Использование:}
\begin{lstlisting}[language=python, numbers=none, frame=single]
rdc exec [OPTIONS] COMMAND [ARGS]...
\end{lstlisting}

\textbf{Параметры:}
\begin{center}
\fontsize{10pt}{10pt}\selectfont
\begin{longtable}[]{p{5cm}|p{2cm}|p{3.5cm}|p{5cm}}
    \hline
\toprule()
Параметр & Тип & Описание & По умолчанию \\
\hline
\midrule()
\endhead
\texttt{-\/-help} & boolean & показать справку &
\texttt{False} \\
\bottomrule()
\hline
\end{longtable}
\end{center}

\hypertarget{rdc-exec-traj}{%
\subsubsection{ Команда rdc exec traj}\label{rdc-exec-traj}}

\textbf{Использование:}
\begin{lstlisting}[language=python, numbers=none, frame=single]
rdc exec traj [OPTIONS]
\end{lstlisting}

\textbf{Параметры:}
\begin{center}
\fontsize{10pt}{10pt}\selectfont
\begin{longtable}[]{p{5cm}|p{2cm}|p{3.5cm}|p{5cm}}
    \hline
\toprule()
Параметр & Тип & Описание & По умолчанию \\
\hline
\midrule()
\endhead
\texttt{-\/-urdf\_path} & text & путь до urdf модели рообота &
\texttt{urdfs/z1\_ee.urdf} \\
\hline
\texttt{-\/-meshes\_path} & text & путь до mesh моделей робота &
\texttt{urdfs/meshes} \\
\hline
\texttt{-\/-data\_path} & text & путь до директории с траекторией для выполнения \\
\hline
\texttt{-\/-sim} & boolean & если True, скрипт будет запущен в режиме симуляции. & \texttt{False} \\
\hline
\texttt{-r} & boolean & если True, скрипт будет рекурсивно выполняться для каждого найденного файла desired\_trajectory.csv & \texttt{False} \\
\hline
\texttt{-\/-skip\_exist} & boolean & если True, скрипт пропустит папки, где есть записи & \texttt{False} \\
\hline
\texttt{-\/-joint\_num} & text & какой номер сочленения использовать. Работает только для траекторий трения & None \\
\hline
\texttt{-\/-type} & choise (\texttt{nominal} \textbar{} \texttt{slow}
\textbar{} \texttt{warmup}) & тип траектории & \texttt{nominal} \\
\hline
\texttt{-\/-refine\_dt} & boolean & выполнить проверку dt между точками траектории и пересчитать траекторию для более высокого dt & \texttt{True} \\
\hline
\texttt{-\/-check} & boolean & выполнить проверку траектории на самостолкновение, ограничения точек и столкновение с полом перед выполнением &
\texttt{True} \\
\hline
\texttt{-\/-times} & integer диапазон (от \texttt{1} до \texttt{100})
& Число повторений траектории & \texttt{1} \\
\hline
\texttt{-\/-load\_urdf\_path} & text & путь к модели urdf для номинальной нагрузки & None \\
\hline
\texttt{-\/-load\_attached\_frame\_name} & text & звено на роботе для крепления модели номинальной нагрузки& None \\
\hline
\texttt{-\/-help} & boolean & показать справку &
\texttt{False} \\
\bottomrule()
\hline
\end{longtable}
\end{center}

\hypertarget{rdc-gen}{%
\subsection{Команда rdc gen}\label{rdc-gen}}

Генерация траекторий для идентификации

Создание каталогов с csv, содержащих специальные траектории для обратной динамической идентификации модели

\textbf{Использование:}
\begin{lstlisting}[language=python, numbers=none, frame=single]
rdc gen [OPTIONS] COMMAND [ARGS]...
\end{lstlisting}

\textbf{Параметры:}
\begin{center}
\fontsize{10pt}{10pt}\selectfont
\begin{longtable}[]{p{5cm}|p{2cm}|p{3.5cm}|p{5cm}}
    \hline
\toprule()
Параметр & Тип & Описание & По умолчанию \\
\hline
\midrule()
\endhead
\texttt{-\/-help} & boolean & показать справку &
\texttt{False} \\
\bottomrule()
\hline
\end{longtable}
\end{center}

\hypertarget{rdc-gen-fourier}{%
\subsubsection{ Команда rdc gen fourier}\label{rdc-gen-fourier}}

Запустить генерацию оптимальной траектории на основе ряда Фурье для последовательного робота-манипулятора.

Пример использования:
\begin{lstlisting}[language=python, numbers=none, frame=single]
rdc gen fourier -\/-pipeline\_name pipeline\_no\_load -\/-output\_path
logs -\/-save\_plots
\end{lstlisting}

\textbf{Использование:}
\begin{lstlisting}[language=python, numbers=none, frame=single]
rdc gen fourier [OPTIONS]
\end{lstlisting}

\textbf{Параметры:}
\begin{center}
\fontsize{10pt}{10pt}\selectfont
\begin{longtable}[]{p{5cm}|p{2cm}|p{3.5cm}|p{5cm}}
    \hline
\toprule()
Параметр & Тип & Описание & По умолчанию \\
\hline
\midrule()
\endhead
\texttt{-\/-config\_path} & text & путь до config-файла с параметрами генерации. &
\texttt{./configs/fourier\_trajectory\_optimization\_pipelines.yaml} \\
\hline
\texttt{-\/-pipeline\_name} & text & выбрать алгоритм для выполнения. Если не указано выполняет все алгоритмы & `` \\
\hline
\texttt{-\/-output\_path} & text & директория для сохранения сгенерированных траекторий & \texttt{logs} \\
\hline
\texttt{-\/-save\_plots} & boolean & сохранять графики &
\texttt{False} \\
\hline
\texttt{-\/-help} & boolean & показать справку &
\texttt{False} \\
\hline
\bottomrule()
\hline
\end{longtable}
\end{center}

\hypertarget{rdc-gen-friction}{%
\subsubsection{ Команда rdc gen friction}\label{rdc-gen-friction}}

Сгенерировать траектории для идентиикации трения.

\textbf{Использование:}
\begin{lstlisting}[language=python, numbers=none, frame=single]
rdc gen friction [OPTIONS]
\end{lstlisting}

\textbf{Параметры:}
\begin{center}
\fontsize{10pt}{10pt}\selectfont
\begin{longtable}[]{p{5cm}|p{2cm}|p{3.5cm}|p{5cm}}
    \hline
\toprule()
Параметр & Тип & Описание & По умолчанию \\
\hline
\midrule()
\endhead
\texttt{-\/-config\_path} & text & путь до конфига для генерации трения &
\texttt{./configs/z1\_friction\_gen\_traj\_config.yaml} \\
\hline
\texttt{-\/-urdf\_path} & text & путь до urdf модели робота&
\texttt{./urdfs/z1\_ee.urdf} \\
\hline
\texttt{-\/-output\_path} & text & путь для сохранения траекторий &
\texttt{./data/friction} \\
\hline
\texttt{-\/-skip\_exist} & boolean & если True, директории, где ...desired\_trajectory.csv уже существуют будут промущены & \texttt{False} \\
\hline
\texttt{-\/-type} & text & тип генерируемой траектории & friction \\
\hline
\texttt{-\/-help} & boolean & показать справку &
\texttt{False} \\
\bottomrule()
\hline
\end{longtable}
\end{center}

\hypertarget{rdc-ident}{%
\subsection{Команда rdc ident}\label{rdc-ident}}

Идентификация модели обратной динамики.

Проидентифицировать полную модель или только модель трения на собранном датасете. В результате будет получена \textit{.rdcm} модель

\textbf{Использование:}
\begin{lstlisting}[language=python, numbers=none, frame=single]
rdc ident [OPTIONS] COMMAND [ARGS]...
\end{lstlisting}

\textbf{Параметры:}
\begin{center}
\fontsize{10pt}{10pt}\selectfont
\begin{longtable}[]{p{5cm}|p{2cm}|p{3.5cm}|p{5cm}}
    \hline
\toprule()
Параметр & Тип & Описание & По умолчанию \\
\hline
\midrule()
\endhead
\texttt{-\/-help} & boolean & показать справку &
\texttt{False} \\
\bottomrule()
\hline
\end{longtable}
\end{center}

\hypertarget{rdc-ident-friction}{%
\subsubsection{ Команда rdc ident friction}\label{rdc-ident-friction}}
Идентификация модели трения
Перед использованием сгенерируйте и выполните специальные LSPB траектории.

\textbf{Использование:}
\begin{lstlisting}[language=python, numbers=none, frame=single]
rdc ident friction [OPTIONS]
\end{lstlisting}

\textbf{Параметры:}
\begin{center}
\fontsize{10pt}{10pt}\selectfont
\begin{longtable}[]{p{5cm}|p{2cm}|p{3.5cm}|p{5cm}}
    \hline
\toprule()
Параметр & Тип & Описание & По умолчанию \\
\hline
\midrule()
\endhead
\texttt{-\/-data\_path} & text & путь для чтения траекторий &
\_required \\
\hline
\texttt{-\/-save\_path} & text & путь до сохранения модели & \_required \\
\hline
\texttt{-\/-offset} & integer & оффсет для плато & \texttt{50} \\
\hline
\texttt{-\/-save\_plots\_path} & text & путь для сохранения графиков & None \\
\hline
\texttt{-\/-help} & boolean & показать справку &
\texttt{False} \\
\hline
\bottomrule()
\hline
\end{longtable}
\end{center}

\hypertarget{rdc-ident-full}{%
\subsubsection{ Команда rdc ident full}\label{rdc-ident-full}}

Идентификация полной модели обратной динамики

Перед использованием необходимо сгенерировать и выполнить траектории для сбора датасета.

\textbf{Использование:}
\begin{lstlisting}[language=python, numbers=none, frame=single]
rdc ident full [OPTIONS]
\end{lstlisting}

\textbf{Параметры:}
\begin{center}
\fontsize{10pt}{10pt}\selectfont
\begin{longtable}[]{p{5cm}|p{2cm}|p{3.5cm}|p{5cm}}
    \hline
\toprule()
Параметр & Тип & Описание & По умолчанию \\
\hline
\midrule()
\endhead
\texttt{-\/-config\_path} & text & путь до файла config-файла &
\texttt{configs/learning\_config.yaml} \\
\hline
\texttt{-\/-urdf\_path} & text & путь до urdf файла робота &
\texttt{urdfs/z1\_ee.urdf} \\
\hline
\texttt{-\/-save\_path} & text & путь, куда необходимо сохранить .rdcm модель &
\texttt{.} \\
\hline
\texttt{-\/-model\_name} & text & название модели & \_required \\
\hline
\texttt{-\/-data\_path} & text & Path to directory containing
recorded\_trajectory.csv & \_required \\
\hline
\texttt{-r}, \texttt{-\/-recursive} / \texttt{-nr},
\texttt{-\/-no\_recursive} & boolean & не искать csv датасетов рекурсивно &
\texttt{True} \\
\hline
\texttt{-\/-save\_dataset} / \texttt{-\/-no\_save\_dataset} & boolean &
путь до сохранения полного датасета, полученного путем агрегированния при рекурсивном поиске& \texttt{True} \\
\hline
\texttt{-\/-accel\_mode} & choice (\texttt{data} \textbar{}
\texttt{estimate} \textbar{} \texttt{off}) & как получить ускорения: data (напрямую из датасета), estimate (путем численного дифференцирования) или off (не использовать ускорения, принудительно использовать фильтрованную динамику) & \texttt{estimate} \\
\hline
\texttt{-\/-train\_ratio} & float range (between \texttt{0} and
\texttt{1}) & соотношение для разделения датасета на train и test & \texttt{0.75} \\
\hline
\texttt{-\/-plot}, \texttt{-p} / \texttt{-\/-no\_plot}, \texttt{-np} &
boolean & выводить графики результатов обучения & \texttt{True} \\
\hline
\texttt{-v}, \texttt{-\/-verbose} / \texttt{-q}, \texttt{-\/-quiet} &
boolean & выводить служебную информацию & \texttt{True} \\
\hline
\texttt{-\/-help} & boolean & показать справку &
\texttt{False} \\
\bottomrule()
\hline
\end{longtable}
\end{center}

\hypertarget{rdc-verification}{%
\subsection{Команда rdc verification}\label{rdc-verification}}

Верификаци модели обратной динамики.

Верификация модели обратной динамики методом вычисленных моментов или по оценке пространственных сил и моментов на рабочем инструменте манипулятора.

\textbf{Использование:}
\begin{lstlisting}[language=python, numbers=none, frame=single]
rdc verification [OPTIONS] COMMAND [ARGS]...
\end{lstlisting}

\textbf{Параметры:}
\begin{center}
\fontsize{10pt}{10pt}\selectfont
\begin{longtable}[]{p{5cm}|p{2cm}|p{3.5cm}|p{5cm}}
    \hline
\toprule()
Параметр & Тип & Описание & По умолчанию \\
\hline
\midrule()
\endhead
\texttt{-\/-help} & boolean & показать справку &
\texttt{False} \\
\bottomrule()
\hline
\end{longtable}
\end{center}

\hypertarget{rdc-verification-torques}{%
\subsubsection{ Команда rdc verification torques}\label{rdc-verification-torques}}

\textbf{Использование:}
\begin{lstlisting}[language=python, numbers=none, frame=single]
rdc verification torques [OPTIONS]
\end{lstlisting}

\textbf{Параметры:}
\begin{center}
\fontsize{10pt}{10pt}\selectfont
\begin{longtable}[]{p{5cm}|p{2cm}|p{3.5cm}|p{5cm}}
    \hline
\toprule()
Параметр & Тип & Описание & По умолчанию \\
\hline
\midrule()
\endhead
\texttt{-\/-traj\_dir\_path} & text & N/A &
\texttt{./data/base\_trj/opttraj\_0\_85cond\_7h\_2000Hz/recorded\_\_19\_11\_2024\_18\_13\_56} \\
\hline
\texttt{-\/-roll\_angle} & float & угол наклона платформы вокруг оси x & \texttt{0.0} \\
\hline
\texttt{-\/-pitch\_angle} & float & угол наклона платформы вокруг оси y & \texttt{0.0} \\
\hline
\texttt{-\/-rdc\_path} & text & путь до .rdcm модели &
\texttt{{[}\textquotesingle{}full.rdcm\textquotesingle{},\ \textquotesingle{}rdc\_model\_new.rdcm\textquotesingle{}{]}} \\
\hline
\texttt{-\/-help} & boolean & показать справку &
\texttt{False} \\
\bottomrule()
\hline
\end{longtable}
\end{center}

\hypertarget{rdc-verification-wrenches}{%
\subsubsection{ Команда rdc verification wrenches}\label{rdc-verification-wrenches}}

\textbf{Использование:}
\begin{lstlisting}[language=python, numbers=none, frame=single]
rdc verification wrenches [OPTIONS]
\end{lstlisting}

\textbf{Параметры:}
\begin{center}
\fontsize{10pt}{10pt}\selectfont
\begin{longtable}[]{p{5cm}|p{2cm}|p{3.5cm}|p{5cm}}
    \hline
\toprule()
Параметр & Тип & Описание & По умолчанию \\
\hline
\midrule()
\endhead
\texttt{-\/-urdf\_system} & text & путь до urdf робота с инструментом
&
\texttt{./urdfs/z1\_tool.urdf} \\
\hline
\texttt{-\/-rdc\_path} & text & путь до .rdcm модели &
\texttt{rdc\_model\_new.rdcm} \\
\hline
\texttt{-\/-num\_poses} & integer & колличество положений для сбора датасета &
\texttt{1} \\
\hline
\texttt{-\/-tool\_link\_name} & text & название СК конца рабочего инструмента &
\texttt{tool\_end\_link} \\
\hline
\texttt{-\/-ft\_transform\_urdf} & text & urdf с СК ft датчика &
\texttt{./urdfs/sensor.urdf} \\
\hline
\texttt{-\/-ft\_link\_name} & text & название СК датчика, где измеряются силы &
\texttt{sensor\_link} \\
\hline
\texttt{-\/-roll\_angle} & float & угол наклона платформы вокруг оси x & \texttt{0.0} \\
\hline
\texttt{-\/-pitch\_angle} & float & угол наклона платформы вокруг оси y & \texttt{0.0} \\
\hline
\texttt{-\/-save\_data} & boolean & сохранять или нет собранные данные (будет создан csv файл) & \texttt{False} \\
\hline
\texttt{-\/-approaching\_speed} & float & скорость надавливания в точке & \texttt{0.005} \\
\hline
\texttt{-\/-max\_force} & float & максимальная сила, прикладываемая к ft датчику &
\texttt{60.0} \\
\hline
\texttt{-\/-help} & boolean & показать справку &
\texttt{False} \\
\bottomrule()
\hline
\end{longtable}
\end{center}
