% \newpage\annotation
% \addcontentsline{toc}{section}{АННОТАЦИЯ}
\newpage\annotation
\addcontentsline{toc}{section}{АННОТАЦИЯ}


%Данный документ является примером оформления текста с использованием системы верстки (La)\TeX. Ссылка: \url{https://en.wikipedia.org/wiki/LaTeX}. Отличительной чертой проекта, намного повышающей удобство использования, является использование файла UseLatex.cmake, который позволяет легко и просто собирать исходные тексты из *.tex файлов путем написания соответствующего CMakeLists.txt (пример имеется в директории проекта) и вызова процесса сборки стандартным способом: \lstinline|mkdir build && cd build && cmake .. && make|.
%
%Доработанный класс espd.cls позволяет легко и просто оформлять титульную страницу и лист утверждения по ГОСТ-19, а также включает все необходимое оформление. Таким образом, использование данного класса и языка разметки (La)\TeX~позволяет техническому писателю сконцентрироваться на главном\mdash написании текста. Оформление формул, таблиц, вставка рисунков также значительно упрощаются, исключается их <<съезжание>>, как часто случается при исползовании текстового редактора Word, особенно разных версий.

% Далее изложены наиболее часто встречающиеся конструкции, необходимые для написания текста технического задания и остальной документации по ГОСТ\sdash 19. \cite{gost19106}

% \subsection{Общие сведения}

% \programname (далее программа \productname) представляет собой контейнеризированный программный пакет, запускаемый посредством развертывания с использованием инструмента  helm.
Программа <<RDC Toolbox>> представляет собой Python-пакет, состоящий из набора библиотек \textit{RDC Toolbox} и набора команд \textit{RDC Tools} для интерфейса командной строки.\\

Структура программы:
\begin{enumerate}
    \item \verb|./src/rdc_toolbox/|~--- содержит файлы библиотеки для синтеза и обучения моделей обратной динамики роботов;
    \item \verb|./src/rdc_tools/|~--- содержит модули программы <<RDC Toolbox>>, которые являются исполняемыми командами;
    \begin{enumerate}
        \item \verb|calibration/sensor_calibration.py|~--- заголовочный файл для работы с матрицами;
        \item \verb|calibration/tool_calibration.py|~--- заголовочный файл для работы с матрицами;

        \item \verb|verification/fs_verification.py|~--- исходный код для работы с матрицами;
        \item \verb|verification/torques_verification.py|~--- исходный код для работы с матрицами;
        
        \item \verb|execution/exec_trajectory.py|~--- исходный код для работы с матрицами;
        
        \item \verb|generation/gen_trajectory_fourier.py|~--- исходный код для работы с матрицами;
        \item \verb|generation/gen_trajectory_friction.py|~--- исходный код для работы с матрицами;

        \item \verb|identification/friction.py|~--- исходный код для работы с матрицами;
        \item \verb|identification/full.py|~--- исходный код для работы с матрицами;
    \end{enumerate}
\end{enumerate}


Настоящий документ подготовлен в соответствии с требованиями ГОСТ\sdash 19.402\sdash 78~\cite{gost19106}.
