% \newpage\annotation
% \addcontentsline{toc}{section}{АННОТАЦИЯ}
\newpage\annotation
\addcontentsline{toc}{section}{АННОТАЦИЯ}


%Данный документ является примером оформления текста с использованием системы верстки (La)\TeX. Ссылка: \url{https://en.wikipedia.org/wiki/LaTeX}. Отличительной чертой проекта, намного повышающей удобство использования, является использование файла UseLatex.cmake, который позволяет легко и просто собирать исходные тексты из *.tex файлов путем написания соответствующего CMakeLists.txt (пример имеется в директории проекта) и вызова процесса сборки стандартным способом: \lstinline|mkdir build && cd build && cmake .. && make|.
%
%Доработанный класс espd.cls позволяет легко и просто оформлять титульную страницу и лист утверждения по ГОСТ-19, а также включает все необходимое оформление. Таким образом, использование данного класса и языка разметки (La)\TeX~позволяет техническому писателю сконцентрироваться на главном\mdash написании текста. Оформление формул, таблиц, вставка рисунков также значительно упрощаются, исключается их <<съезжание>>, как часто случается при исползовании текстового редактора Word, особенно разных версий.

% Далее изложены наиболее часто встречающиеся конструкции, необходимые для написания текста технического задания и остальной документации по ГОСТ\sdash 19. \cite{gost19106}

% \subsection{Общие сведения}

% \programname (далее программа \productname) представляет собой контейнеризированный программный пакет, запускаемый посредством развертывания с использованием инструмента  helm.
Программа самодиагностики датчика положения представляет собой программный пакет, запускаемый на микроконтроллере.\\
Структура программы:
\begin{enumerate}
    \item \verb|./mcu-libs/|~--- содержит файлы библиотеки для работы с матрицами на микроконтроллере и оценивания с помощью рекурентного метода наименьших квадратов;
    \begin{enumerate}
        \item \verb|Matrix.h|~--- заголовочный файл для работы с матрицами;
        \item \verb|Matrix.cpp|~--- исходный код для работы с матрицами;
        \item \verb|RLS.h|~--- заголовочный файл для рекуррентного метода наименьших квадратов;
        \item \verb|RLS.cpp|~--- исходный код рекуррентного метода наименьших квадратов;
    \end{enumerate}
    \item \verb|./dsp_rfft_windowed.h|~--- заголовочный файл с определением функций алгоритма быстрого преобразования Фурье с использованием DSP-ядра микроконтроллера;
    \item \verb|./dsp_rfft_windowed.cpp|~--- исходный код с реализацией алгоритма быстрого преобразования Фурье с использованием DSP-ядра микроконтроллера;
    \item \verb|./dsp_setup.h|~--- заголовочный файл с определением функций инициализации DSP-ядра микроконтроллера;
    \item \verb|./dsp_setup.cpp|~--- исходный код с инициализацией DSP-ядра, а также установлением тактовой частоты, настройки прерываний и аллокации памяти для микроконтроллера;
    \item \verb|./sensor_self_diagnostic.h|~--- заголовочный файл с определением функций проверки несоосности, превышения допустимого воздушного зазора между статором и ротором, а также проверки некорректного монтажа датчика и наличия других внутренних ошибок;
    \item \verb|./sensor_self_diagnostic.cpp|~--- исходный код с реализацией алгоритмов самодиагностики датчика положения:
    \begin{enumerate}
        \item Функция \verb|void fullInitCheck(...)|~--- предназначена для выполнения первичной самодиагностики, а именно опеределения превышения допустимого воздушного зазора и неверный монтаж датчика положения.
        \item Функция \verb|void fullRuntimeCheck(...)|~--- предназначена для выполнения самодиагностики в режиме реального времени, а именно опеределения несоосности ротора и статора.
    \end{enumerate}
    % \item \verb|CMakeLists.txt|~--- файл для сборки программы на персональном компьютере;
    % \item \verb|docker-compose.yaml|~--- файл для автоматического запуска проверок кода санитайзерами;
    % \item \verb|Dockerfile|~--- содержит описание докер-образа с необходимым программным обеспечением для тестирования программы на персональном компьютере;
\end{enumerate}


Настоящий документ подготовлен в соответствии с требованиями ГОСТ\sdash 19.402\sdash 78~\cite{gost19106}.
