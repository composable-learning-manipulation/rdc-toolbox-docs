\begin{lstlisting}[language=python, caption=\raggedright{calibration/sensor\_calibration.py}, frame=single]
import click
import numpy as np
from rdc_tools.utils import URDF_PATH


def create_urdf_from_transform(
    T_matrix: np.ndarray,
    robot_name: str = "calibrated_robot",
    base_link_name: str = "base_link",
    tool_link_name: str = "tool_link",
    joint_name: str = "base_to_tool_joint",
    urdf_filename: str = "robot.urdf",
    tool_visual_mesh: str = None,
    tool_collision_mesh: str = None,
    mass: float = 1.0,
    inertia: dict = None,
):
    """
    Creates a new URDF file defining a robot with a base link and a tool link
    connected via a fixed joint, based on the provided transformation matrix.

    Parameters:
    - T_matrix (np.ndarray): 4x4 homogeneous transformation matrix.
    - robot_name (str): Name of the robot.
    - base_link_name (str): Name of the base link.
    - tool_link_name (str): Name of the tool link.
    - joint_name (str): Name of the joint connecting base to tool.
    - urdf_filename (str): Filename to save the URDF.
    - tool_visual_mesh (str): Path to visual mesh file for tool link.
    - tool_collision_mesh (str): Path to collision mesh file for tool link.
    - mass (float): Mass of the tool link in kg.
    - inertia (dict): Inertia tensor components with keys 'ixx', 'ixy', 'ixz',
                        'iyy', 'iyz', 'izz'. Defaults to unit inertia.
    """

    # Validate T_matrix
    if T_matrix.shape != (4, 4):
        raise ValueError("T_matrix must be a 4x4 numpy array.")

    # Decompose T_matrix into rotation and translation
    rotation_matrix = T_matrix[:3, :3]
    translation = T_matrix[:3, 3]

    # Convert rotation matrix to RPY angles
    r = R.from_matrix(rotation_matrix)
    rpy = r.as_euler("xyz", degrees=False)  # RPY in degrees for URDF

    # Set default inertia if not provided
    if inertia is None:
        inertia = {
            "ixx": "0.1",
            "ixy": "0.0",
            "ixz": "0.0",
            "iyy": "0.1",
            "iyz": "0.0",
            "izz": "0.1",
        }
    else:
        # Ensure all inertia components are provided
        required_keys = ["ixx", "ixy", "ixz", "iyy", "iyz", "izz"]
        for key in required_keys:
            if key not in inertia:
                raise ValueError(f"Inertia tensor missing '{key}' component.")

    # Create the root <robot> element
    robot = ET.Element("robot", name=robot_name)

    # Define the base link
    base_link = ET.SubElement(robot, "link", name=base_link_name)

    # Add inertial element to base link (optional, can be customized)
    base_inertial = ET.SubElement(base_link, "inertial")
    ET.SubElement(base_inertial, "origin", xyz="0 0 0", rpy="0 0 0")
    ET.SubElement(base_inertial, "mass", value="0.0")  # Massless base
    ET.SubElement(
        base_inertial,
        "inertia",
        ixx="0.0",
        ixy="0.0",
        ixz="0.0",
        iyy="0.0",
        iyz="0.0",
        izz="0.0",
    )

    # Define the tool link
    tool_link = ET.SubElement(robot, "link", name=tool_link_name)

    # Add inertial element to tool link
    tool_inertial = ET.SubElement(tool_link, "inertial")
    ET.SubElement(tool_inertial, "origin", xyz="0 0 0", rpy="0 0 0")
    ET.SubElement(tool_inertial, "mass", value=str(mass))
    ET.SubElement(tool_inertial, "inertia", **inertia)

    # Add visual element to tool link
    tool_visual = ET.SubElement(tool_link, "visual")
    ET.SubElement(tool_visual, "origin", xyz="0 0 0", rpy="0 0 0")
    if tool_visual_mesh and os.path.isfile(tool_visual_mesh):
        ET.SubElement(tool_visual, "geometry").append(
            ET.Element("mesh", filename=tool_visual_mesh)
        )
    else:
        # Default to a simple box if no mesh provided
        geometry = ET.SubElement(tool_visual, "geometry")
        ET.SubElement(geometry, "box", size="0.05 0.05 0.05")  # 5cm cube

    # Add collision element to tool link
    tool_collision = ET.SubElement(tool_link, "collision")
    ET.SubElement(tool_collision, "origin", xyz="0 0 0", rpy="0 0 0")
    if tool_collision_mesh and os.path.isfile(tool_collision_mesh):
        ET.SubElement(tool_collision, "geometry").append(
            ET.Element("mesh", filename=tool_collision_mesh)
        )
    else:
        # Default to a simple box if no mesh provided
        geometry = ET.SubElement(tool_collision, "geometry")
        ET.SubElement(geometry, "box", size="0.05 0.05 0.05")  # 5cm cube

    # Define the joint connecting base to tool
    joint = ET.SubElement(robot, "joint", name=joint_name, type="fixed")
    ET.SubElement(joint, "parent", link=base_link_name)
    ET.SubElement(joint, "child", link=tool_link_name)
    ET.SubElement(
        joint,
        "origin",
        xyz=f"{translation[0]:.6f} {translation[1]:.6f} {translation[2]:.6f}",
        rpy=f"{rpy[0]:.6f} {rpy[1]:.6f} {rpy[2]:.6f}",
    )

    # Create the ElementTree and write to file
    tree = ET.ElementTree(robot)
    ET.indent(tree, space="  ", level=0)  # For pretty-printing (Python 3.9+)
    tree.write(urdf_filename, encoding="utf-8", xml_declaration=True)

    print(f"URDF file '{urdf_filename}' has been created successfully.")


def inverse_kinematics(model, data, pdes):
    JOINT_ID = 6
    oMdes = pinocchio.SE3(np.eye(3), np.array(pdes))

    q = pinocchio.neutral(model)
    eps = 1e-3
    IT_MAX = 1000
    DT = 1e-1
    damp = 1e-12

    i = 0
    while True:
        pinocchio.forwardKinematics(model, data, q)
        dMi = oMdes.actInv(data.oMi[JOINT_ID])
        err = pinocchio.log(dMi).vector
        if np.linalg.norm(err) < eps:
            success = True
            break
        if i >= IT_MAX:
            success = False
            break
        J = pinocchio.computeJointJacobian(model, data, q, JOINT_ID)
        v = -J.T.dot(np.linalg.solve(J.dot(J.T) + damp * np.eye(6), err))
        q = pinocchio.integrate(model, q, v * DT)
        # if not i % 10:
        #     print("%d: error = %s" % (i, err.T))
        i += 1
    # print(f"{pdes = }")
    if not success:
        raise RuntimeError(
            f"The iterative algorithm has not reached convergence to the desired precision, {np.linalg.norm(err) = }"
        )
    return q


class SensorCalibration:
    def __init__(
        self,
        urdf_filename,
    ):
        self._model = pinocchio.buildModelFromUrdf(urdf_filename)
        self._data_for_model = self._model.createData()
        self._data_storage_holes = {i: [] for i in range(5)}
        self._data_storage_surface = []

        for idx, frame in enumerate(self._model.frames):
            print(f"Index: {idx}, Name: {frame.name}")
        try:
            self._eef_i = self._model.getFrameId("tool_end_link")
        except pinocchio.FrameNameException:
            raise ValueError("Frame 'link06' not found in the model.")

        print(
            f"EEF frame is {self._eef_i } -- {self._model.frames[self._eef_i].name}"
        )

    def add_data_holes(self, position_index, q):
        pinocchio.framesForwardKinematics(self._model, self._data_for_model, q)
        self._data_storage_holes[position_index].append(
            self._data_for_model.oMf[self._eef_i].translation.copy()
        )

    def add_data_surface(self, q):
        pinocchio.framesForwardKinematics(self._model, self._data_for_model, q)
        self._data_storage_surface.append(
            self._data_for_model.oMf[self._eef_i].translation.copy()
        )

    def _get_surface_normal_lstsq(self):
        points = np.array(self._data_storage_surface).reshape((-1, 3))
        x = points[:, 0]
        y = points[:, 1]
        z = points[:, 2]
        ones = np.ones_like(x)
        C = 1
        A, B, D = np.linalg.lstsq(
            np.column_stack((x, y, ones)), -z, rcond=None
        )[0]
        centroid = np.mean(points, axis=0)
        normal_vector = np.array([A, B, C])
        return normal_vector / np.linalg.norm(normal_vector), centroid

    def _project_point_onto_plane(self, Q, P0, normal_vector):
        Q = np.array(Q)
        P0 = np.array(P0)
        n = np.array(normal_vector)
        n_norm_sq = np.dot(n, n)
        if n_norm_sq == 0:
            raise ValueError("Normal vector cannot be zero.")
        vector = Q - P0
        distance = np.dot(vector, n) / n_norm_sq
        Q_prime = Q - distance * n
        return Q_prime

    def calib_results(self):
        surface_normal, centroid = self._get_surface_normal_lstsq()
        points = [
            np.mean(
                np.array(self._data_storage_holes[i]).reshape(-1, 3), axis=0
            )
            for i in range(5)
        ]
        points = [
            self._project_point_onto_plane(point, centroid, surface_normal)
            for point in points
        ]
        center = np.mean(points[1:], axis=0)
        vx = points[0] - center
        vx /= np.linalg.norm(vx)
        # vz = surface_normal
        vy = (points[1] + points[2]) * 0.5 - center
        vy /= np.linalg.norm(vy)
        vz = np.cross(vx, vy)
        vz /= np.linalg.norm(vz)
        print(f"vx dot vy before {np.dot(vx.flatten(), vy.flatten())}")
        # ISSUE!!!!!!!!!
        vy = np.cross(vz, vx)
        vy /= np.linalg.norm(vy)

        vx = vx.reshape(-1, 1)
        vy = vy.reshape(-1, 1)
        vz = vz.reshape(-1, 1)
        center = center.reshape(-1, 1)

        assert np.isclose(
            np.dot(vx.flatten(), vy.flatten()), 0
        ), f"vx and vy are not orthogonal, {np.dot(vx.flatten(), vy.flatten()) = }"
        assert np.isclose(np.linalg.norm(vx), 1), "vx is not normalized"
        assert np.isclose(np.linalg.norm(vy), 1), "vy is not normalized"
        assert np.isclose(np.linalg.norm(vz), 1), "vz is not normalized"

        T_matrix = np.hstack([vx, vy, vz, center])
        T_matrix = np.vstack([T_matrix, [0, 0, 0, 1]])
        return T_matrix


def add_tool_link_to_urdf(
    original_urdf,
    updated_urdf,
    T_matrix,
    tool_link_name="tool_link",
    tool_joint_name="tool_joint",
    parent_link="link06",
):
    """
    Adds a new tool link to the URDF based on the calibration transformation matrix.

    Parameters:
    - original_urdf (str): Path to the original URDF file.
    - updated_urdf (str): Path to save the updated URDF file.
    - T_matrix (np.ndarray): 4x4 transformation matrix from calibration.
    - tool_link_name (str): Name of the new tool link.
    - tool_joint_name (str): Name of the new joint connecting the tool link.
    - parent_link (str): Name of the parent link to which the tool is attached.
    """
    # Parse the original URDF
    tree = ET.parse(original_urdf)
    root = tree.getroot()

    # Check if the tool_link already exists
    existing_links = [link.get("name") for link in root.findall("link")]
    if tool_link_name in existing_links:
        print(f"Link '{tool_link_name}' already exists in the URDF.")
        return

    # Create the new tool link
    tool_link = ET.Element("link", {"name": tool_link_name})

    root.append(tool_link)
    print(f"Added new link '{tool_link_name}' to the URDF.")

    joint = ET.Element(
        "joint",
        {
            "name": tool_joint_name,
            "type": "fixed",
        },
    )

    parent = ET.SubElement(joint, "parent", {"link": parent_link})
    child = ET.SubElement(joint, "child", {"link": tool_link_name})

    rotation_matrix = T_matrix[:3, :3]
    translation = T_matrix[:3, 3]

    rpy = R.from_matrix(rotation_matrix).as_euler("xyz", degrees=False)

    # Create the origin element with translation and rotation
    origin = ET.SubElement(
        joint,
        "origin",
        {
            "xyz": f"{translation[0]:.6f} {translation[1]:.6f} {translation[2]:.6f}",
            "rpy": f"{rpy[0]:.6f} {rpy[1]:.6f} {rpy[2]:.6f}",
        },
    )

    root.append(joint)
    print(
        f"Added new joint '{tool_joint_name}' connecting '{parent_link}' to '{tool_link_name}'."
    )
    tree.write(updated_urdf)
    print(f"Updated URDF saved to '{updated_urdf}'.")


@click.command(name="ftsensor")
@click.option(
    "--urdf_z1",
    default=URDF_PATH + "/z1_ee.urdf",
    help="Original Robots urdf path",
)
@click.option(
    "--urdf_z1_link_name",
    default="link_ee",
    help="Z1 urdf ee link_name",
)
@click.option(
    "--urdf_tool",
    default=URDF_PATH + "/tool_calib.urdf",
    help="Tool urdf path",
)
@click.option(
    "--urdf_tool_link_name",
    default="tool_start_link",
    help="Tool urdf start link link_name",
)
@click.option(
    "--upd_urdf",
    default=URDF_PATH + "/z1_tool.urdf",
    help="Updated urdf path",
)
@click.option(
    "--urdf_sensor",
    default=URDF_PATH + "/sensor.urdf",
    help="Updated urdf with sensor path",
)
@click.option(
    "--pitch_deg",
    default=0,
    type=float,
    help="Updated urdf with sensor path",
)
@click.option(
    "--roll_deg",
    default=0,
    type=float,
    help="Updated urdf with sensor path",
)
@click.option("--synth", default=False, is_flag=True, help="Synth data flag")
def sensor_calibration(
    urdf_z1,
    urdf_z1_link_name,
    urdf_tool,
    urdf_tool_link_name,
    upd_urdf,
    urdf_sensor,
    pitch_deg,
    roll_deg,
    synth,
):
    """Calibrating the force-sensor TF

    The program can be used to estimate
    the origin and rotation of the force sensor
    corresponding to the world coordinate system.
    """
    global np, ET, R, pinocchio, merge_urdfs_with_fixed_joint, URDF_PATH
    import numpy as np
    import xml.etree.ElementTree as ET
    from scipy.spatial.transform import Rotation as R
    import pinocchio
    from rdc_toolbox.utils.pinocchio import merge_urdfs_with_fixed_joint
    from rdc_tools.calibrations.platform_kinematics import PlatformKinematics

    merge_urdfs_with_fixed_joint(
        urdf_z1,
        urdf_tool,
        urdf_z1_link_name,
        urdf_tool_link_name,
        upd_urdf,
    )
    if synth:
        model = pinocchio.buildModelFromUrdf(upd_urdf)
        data = model.createData()

        x_max = 0.3
        x_min = 0.25
        y_max = 0.32
        y_min = 0.28
        z_height = 0
        p0 = inverse_kinematics(
            model,
            data,
            [
                x_max + 0.01,
                (y_max + y_min) * 0.5,
                z_height + (np.random.rand() - 0.5) * 5 / 1000,
            ],
        )
        p1 = inverse_kinematics(
            model,
            data,
            [x_max, y_max, z_height + (np.random.rand() - 0.5) * 5 / 1000],
        )
        p2 = inverse_kinematics(
            model,
            data,
            [x_min, y_max, z_height + (np.random.rand() - 0.5) * 5 / 1000],
        )
        p3 = inverse_kinematics(
            model,
            data,
            [x_min, y_min, z_height + (np.random.rand() - 0.5) * 5 / 1000],
        )
        p4 = inverse_kinematics(
            model,
            data,
            [x_max, y_min, z_height + (np.random.rand() - 0.5) * 5 / 1000],
        )
        qs = [p0, p1, p2, p3, p4]

        qs_surface = [
            inverse_kinematics(
                model,
                data,
                [
                    0.275 + (np.random.rand() - 0.5) / 10,
                    0.3 + (np.random.rand() - 0.5) / 10,
                    0 + (np.random.rand() - 0.5) * 2 / 1000,
                ],
            )
            for _ in range(10)
        ]
    else:
        from rdc_toolbox.execution.robots import (
            Z1RobotAPI,
        )

        model = pinocchio.buildModelFromUrdf(upd_urdf)
        data = model.createData()
        sim_robot = Z1RobotAPI(gripper=False, pinmodel=model)
        sim_robot.connect()
        sim_robot.teach_on()

    calib = SensorCalibration(upd_urdf)
    print("Go from the top one in anti-clockwise direction")
    for i in range(5):
        print("====================")
        print(f"Go to position {i}.")
        print("   0   \n1     4\n\n2     3")
        while True:
            try:
                if not synth:
                    input(
                        "Press <enter> to record data; ctrl-c to move to the next point."
                    )
                    q = sim_robot.q.numpy()
                    calib.add_data_holes(i, q)
                else:
                    for _ in range(10):
                        calib.add_data_holes(i, qs[i])
                    raise KeyboardInterrupt
            except KeyboardInterrupt as e:
                if i < 4:
                    print("Move to next point")
                else:
                    print("Holes data collection finished")
                break

    print("Collect 10 data point from flat surface of the sensor")
    for i in range(10):
        if not synth:
            input("Press <enter> to record data")
            q = sim_robot.q.numpy()
            calib.add_data_surface(q)
        else:
            calib.add_data_surface(qs_surface[i])

    print("TF matrix to force_sensor:")
    T_matrix = calib.calib_results()

    pk = PlatformKinematics()
    create_urdf_from_transform(
        T_matrix=pk.forward_kinematics(pitch_deg, roll_deg) @ T_matrix,
        robot_name="calibrated_sensor",
        base_link_name="platform_base",
        tool_link_name="sensor_link",
        joint_name="platform_base_to_sensor_link_joint",
        urdf_filename=urdf_sensor,
        tool_visual_mesh=None,  # Provide path if available
        tool_collision_mesh=None,  # Provide path if available
        mass=1.0,
        inertia={
            "ixx": "0.01",
            "ixy": "0.0",
            "ixz": "0.0",
            "iyy": "0.01",
            "iyz": "0.0",
            "izz": "0.01",
        },
    )


if __name__ == "__main__":
    sensor_calibration()
    
\end{lstlisting}