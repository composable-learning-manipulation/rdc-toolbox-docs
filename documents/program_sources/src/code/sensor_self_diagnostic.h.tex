\begin{lstlisting}[caption=\raggedright{mcu-libs/RLS.h}, frame=single]
#pragma once

#include "dsp_rfft_windowed.h"
#include "RLS.h"
#include <math.h>
#include <stdint.h>

#define SD_REQUIRED_BUFF_SIZE 3000U

struct SelfDiagnosticInputState
{
    // Current (alpha, beta):
    // i_a (k) = i_ab[0 + i], i_b (k) = i_ab[SELF_DIAGNOSTIC_REQUIRED_BUFF_SIZE + k]
    float i_ab[2*SD_REQUIRED_BUFF_SIZE] = {0.};
    // Derivative of the current (alpha, beta):
    // di_a (k) = di_ab[0 + i], di_b (k) = di_ab[SELF_DIAGNOSTIC_REQUIRED_BUFF_SIZE + k]
    float di_ab[2*SD_REQUIRED_BUFF_SIZE] = {0.};
    // Voltage (alpha, beta):
    // u_a (k) = u_ab[0 + i], u_b (k) = u_ab[SELF_DIAGNOSTIC_REQUIRED_BUFF_SIZE + k]
    float u_ab[2*SD_REQUIRED_BUFF_SIZE] = {0.};

    //    Magnet torque from digital twin
    float magnet_torque_est[SD_REQUIRED_BUFF_SIZE] = {0.};

    // Data from the sensors (Measurements preprocessor)
    float w[SD_REQUIRED_BUFF_SIZE] = {0.};
    float theta[SD_REQUIRED_BUFF_SIZE] = {0.};

    // Data from velocity-position estimation program
    float w_est[SD_REQUIRED_BUFF_SIZE] = {0.};
    float theta_est[SD_REQUIRED_BUFF_SIZE] = {0.};
};

struct SelfDiagnosticParams
{
    float maxAirGap = 1e-3;
    float maxHighFreqTorqueMagnitude = 5.;
    float wMin = 10 * M_PI;
    float wMax = 20 * M_PI;
    uint16_t coilNTurns = 1000;
    float mu = 135.0;                  // Magnetic permeability of the stator core
    const float mu0 = 4 * M_PI * 1e-7; // Magnetic permeability of vacuum
    float statorCoreArea = 0.0001247;
    float statorWireLen = 9.673;
    float magnetCoreArea = 0.001824;
    uint16_t n_p = 5;
    const float dt = 1e-3; // Period for FFT detecting Shaft vibration
};

enum SelfDiagnosticCode
{
    OK = 0, // No Errors
    SENSOR_MOUNT_ERROR = 1 << 0,
    SHAFT_VIBRATION_ERROR = 1 << 1,
    EXCESS_AIR_GAP_ERROR = 1 << 2,
    MCU_ERROR = 1 << 3,
    INTERNAL_ERROR = MCU_ERROR
};

void fullInitCheck(SelfDiagnosticInputState &input_state, uint16_t &self_diagnostic_state, const SelfDiagnosticParams &diagnostic_params);
void fullRuntimeCheck(SelfDiagnosticInputState &input_state, uint16_t &self_diagnostic_tate, const SelfDiagnosticParams &diagnostic_params);

SelfDiagnosticCode checkSensorMount(SelfDiagnosticInputState &input_state);
SelfDiagnosticCode checkExcessAirGap(SelfDiagnosticInputState &input_state, float &excess_air_gap);
SelfDiagnosticCode FFT_torque_vibration_check(SelfDiagnosticInputState &input_state, float & minFreq, float & maxFreq, float maxMag, float & dt);

\end{lstlisting}